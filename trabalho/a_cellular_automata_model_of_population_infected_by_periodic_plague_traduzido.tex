\documentclass{article}
\usepackage[utf8]{inputenc}

%----------------------------------------------------------------------------------------
% Autor........: Alexandre Gomes da Costa
% Professor....: Marilton Sanchotene de Aguiar
% Disciplina...: Sistemas Evolutivos (SE)
% Descrição....: Trabalho final
%----------------------------------------------------------------------------------------


\begin{document}

%----------------------------------------------------------------------------------------
% Página de Titulo
%----------------------------------------------------------------------------------------

\begin{titlepage}

\newcommand{\HRule}{\rule{\linewidth}{0.5mm}} % Defines a new command for the horizontal lines, change thickness here

\center % Center everything on the page

\textsc{\LARGE Universidade Federal de Pelotas}\\[1.5cm]

\HRule \\[0.4cm]
{\huge \bfseries A Cellular Automata Model of Population Infected by Periodic Plague}\\[0.4cm]
\HRule \\[1.5cm]

\emph{Author:} Alexandre Gomes da Costa

\vfill % Fill the rest of the page with whitespace

\end{titlepage}

%----------------------------------------------------------------------------------------
% Tabela de Conteúdo
%----------------------------------------------------------------------------------------

\tableofcontents % Include a table of contents

\newpage % Begins the essay on a new page instead of on the same page as the table of contents 

%----------------------------------------------------------------------------------------
% Abistract
%----------------------------------------------------------------------------------------

\section{Abstract}

Evolution of a population consisting of individuals, each holding a unique “genetic code”, is modeled on the 2D cellular automata lattice. The “genetic code” represents three episodes of life: the “youth”, the “maturity” and the “old age”. Only the “mature” individuals can procreate. Durations of the life-episodes are variable and are modified due to evolution. We show that the “genetic codes” of individuals self-adapt to environmental conditions in such a way that the entire ensemble has the greatest chance to survive. For a stable environment, the “youth” and the “mature” periods extend extremely during evolution, while the “old age” remains short and insignificant. The unstable environment is modeled by periodic plagues, which attacks the colony. For strong plaques the “young” individuals vanishes while the length of the “old age” period extends. We concluded that while the “maturity” period decides about the reproductive power of the population, the idle life-episodes set up the control mechanisms allowing for self-adaptation of the population to hostile environment. The “youth” accumulates reproductive resources while the “old age” accumulates the space required for reproduction.

Index Terms: cellular automata, population infected.

%----------------------------------------------------------------------------------------
% Resumo
%----------------------------------------------------------------------------------------

\section{Resumo}

Evolução de uma população constituída por indivíduos, cada um guardando um "código genético" único, é inspirado na rede de autômatos celulares 2D. O "código genético" representa três episódios da vida: "Juventude" , "maturidade" e "velhice". Somente as pessoas "maduras" pode procriar. Durações dos episódios de vida são variáveis e são modificados devido à evolução. Nós mostramos que o "código genético" de indivíduos auto-adapta às condições ambientais, de tal maneira que todo o conjunto tem a maior probabilidade de sobreviver. Para um ambiente estável, a "juventude" e os períodos de "maduros" estender extremamente durante a evolução, enquanto que a "velhice" permanece curto e insignificante. O ambiente instável é modelado por pragas periódicas, que ataca a colônia. Para pragas fortes os indivíduos "jovens" desaparece enquanto a duração do período de "velhice" se estende. Concluiu-se que, enquanto o período de "maturidade" decide sobre o poder reprodutivo da população, os episódios da vida ociosa configurar os mecanismos de controle que permitam a auto- adaptação da população ao ambiente hostil. A "juventude" acumula recursos reprodutivos, enquanto a "velhice" acumula o espaço necessário para a reprodução.

Palavras-chave: Automatos Celulares, Evolução de uma população.

%----------------------------------------------------------------------------------------
% Instrodução
%----------------------------------------------------------------------------------------

\section{Introdução}

% The cellular automata paradigm is a perfect computational platform for modeling evolving population. It defines both the communication medium for the agents and the living space. Assuming the lack of individual features, which diversify the population, the modeled system adapts to the unstable environment, developing variety of spatially correlated patterns (see e.g. [1-3]). Formation of patterns of well-defined multi-resolutional structures can be viewed as the result of a complex exchange of information between individuals and the whole population.

O paradigma de autômatos celulares é uma plataforma computacional perfeita para modelar evolução da população. Ela define tanto o meio de comunicação para os agentes como também o espaço de vida. Assumindo a ausência de características individuais, que diversifica a população, o sistema modelado se adapta ao ambiente instável, desenvolvendo variedade de padrões espacialmente correlacionadas (ver por exemplo [1-3]). A formação de padrões de estruturas multi-resolutional bem definidos podem ser vistos como o resultado de uma troca complexa de informação entre indivíduos e toda a população.

% Another type of correlations - correlations in the feature space - emerges for the models of populations in which each individual holds a unique feature vector evolving along with the entire system [4]. The aging is one of the most interesting puzzles of evolution, which can be investigated using this kind of models.

Outro tipo de correlações - Correlações no espaço de características - emerge para os modelos de populações em que cada indivíduo detém um vetor característica única evoluindo junto com todo o sistema [4]. O envelhecimento é um dos quebra-cabeças mais interessantes da evolução, que podem ser investigados usando este tipo de modelos.

% It is widely known that the aging process is mainly determined by the genetic and environmental features. The most of computational models of aging involving genetic factor are based on the famous Penna paradigm [5,6]. This model uses the theory of accumulation, which says that destructive mutations - which consequences depend on the age of individual - can be inherited by the following generations and are accumulated in their genomes. The Penna model suffers from the following important limitations.

É sabido que o processo de envelhecimento é determinada principalmente pelas características genéticas e ambientais. A maioria dos modelos computacionais de envelhecimento envolve factor genético são baseados na famosa paradigma Penna [5,6]. Este modelo utiliza a teoria da acumulação, que diz que as mutações destrutivas - que conseqüências dependem da idade do indivíduo - pode ser herdada pelas gerações seguintes e são acumulados em seus genomas. O modelo Penna sofre das seguintes limitações importantes. 




%----------------------------------------------------------------------------------------
% CA Model of Evolution
%----------------------------------------------------------------------------------------

\section{CA Model of Evolution}



%----------------------------------------------------------------------------------------
% Results of Modeling
%----------------------------------------------------------------------------------------

\section{Results of Modeling}



%----------------------------------------------------------------------------------------
% Concluding Remarks
%----------------------------------------------------------------------------------------

\section{Concluding Remarks}



%----------------------------------------------------------------------------------------
% References
%----------------------------------------------------------------------------------------



%----------------------------------------------------------------------------------------

\end{document}
