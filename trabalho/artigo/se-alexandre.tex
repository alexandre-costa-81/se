\documentclass[12pt]{article}

\usepackage{se-alexandre}

\usepackage{graphicx,url}

\usepackage[brazil]{babel}   
\usepackage[latin1]{inputenc}  

     
\sloppy

\title{Trabalho da discliplina de Sistemas Evolutivos:\\ A Cellular Automata Model of Population Infected by Periodic Plague}

\author{Alexandre G. da Costa\inst{1}}


\address{Centro de Desenvolvimento Tecnol�gico -- Universidade Federal de Pelotas
  (UFPEL)\\
  Pelotas -- RS -- Brasil
  \email{alexandre.costa@inf.ufpel.edu.br}
}

\begin{document} 

% ----------------------------------------------------------------------------
\maketitle

% ----------------------------------------------------------------------------
\begin{abstract}
  This meta-paper describes the style to be used in articles and short papers
  for SBC conferences. For papers in English, you should add just an abstract
  while for the papers in Portuguese, we also ask for an abstract in
  Portuguese (``resumo''). In both cases, abstracts should not have more than
  10 lines and must be in the first page of the paper.
\end{abstract}
  
% ----------------------------------------------------------------------------   
\begin{resumo} 
  Este trabalho descreve a implementa��o do artigo proposto na diciplina de
  Sistemas Evolutivos e prop�e uma nova abordagem da implementa��o. Esse artigo
  descreve um algor�tmo que evolui.
\end{resumo}

% ----------------------------------------------------------------------------
\section{Introdu��o}

Automato celular � um modelos matematicos que foi desenvolvido para simular
a evolu��o natural, por exemplo \textit{Game of Life}. Pois ele define tanto o
meio ambiante como tamb�m os individuos.


% ----------------------------------------------------------------------------
\section{Trabalhos Relacionados}

Nesta sess�o ser� explicado em detalhes o artigo referencial.

O artigo refer�ncial abordado neste trabalho foi \textit{A Cellular Automata 
Model of Population Infected by Periodic Plague} de Witold Dzwinel. 



% ----------------------------------------------------------------------------
\section{Proposta}

Nesta sess�o tem o objetivo de explicar o trabalho proposto, ressaltando as contribui��es da proposta.


% ----------------------------------------------------------------------------
\section{Resultados Alcan�ados}

Se poss�vel comparando com o trabalho referencial.

% ----------------------------------------------------------------------------
\section{Conclus�es}

Apresentar as conclus�es  \cite{dzwinel:04}.

%\begin{figure}[ht]
%\centering
%\includegraphics[width=.5\textwidth]{fig1.jpg}
%\caption{A typical figure}
%\label{fig:exampleFig1}
%\end{figure}

%\begin{figure}[ht]
%\centering
%\includegraphics[width=.3\textwidth]{fig2.jpg}
%\caption{This figure is an example of a figure caption taking more than one
%  line and justified considering margins mentioned in Section~\ref{sec:figs}.}
%\label{fig:exampleFig2}
%\end{figure}

%\begin{table}[ht]
%\centering
%\caption{Variables to be considered on the evaluation of interaction
%  techniques}
%\label{tab:exTable1}
%\includegraphics[width=.7\textwidth]{table.jpg}
%\end{table}

% ----------------------------------------------------------------------------

\bibliographystyle{sbc}
\bibliography{se-alexandre}

\end{document}
